\documentclass{article}
\usepackage{showkeys}
\usepackage{enumerate}
\usepackage[colorlinks=true, urlcolor=blue, linkcolor=red]{hyperref}

% Title page information
\title{
ENFORCEMENT GUIDELINES GOVERNING\\
\textbf{COMPUTER SCIENCE HOUSE'S}\\
SLACK CHAT PLATFORM}
\author{Computer Science House Executive Board}

% Last Modified Date
\newcommand{\datechanged}{Last Updated: \today}
\date{\datechanged}

% Fix margins
\setlength{\evensidemargin}{0in}
\setlength{\oddsidemargin}{0in}
\setlength{\textwidth}{6.5in}
\setlength{\topmargin}{0in}
\setlength{\textheight}{8.5in}

\pagestyle{myheadings}
\markright{{\rm CSH Slack Enforcement Guidelines \hfill \datechanged \hfill Page }}

\begin{document}
\maketitle

The following items are considered to be offenses within Slack that are punishable by the Executive Board.

\begin{enumerate}[I]
\item Anything falling under Title IX

Ex. \href{https://www.rit.edu/fa/compliance/title-ix}{RIT Title IX Office}
\begin{itemize}
    \item \textbf{Emergency Triage}: Message Deletion if necessary (Please screenshot before deletion)
    \item \textbf{All Offenses}: Report to/Handled by Title IX Office
\end{itemize}

\item Unnecessary or unwarranted mass notification

Ex. Any unwarranted mass notification: @here, @channel, @everyone, @active
\begin{itemize}
    \item \textbf{Emergency Triage}: Message Deletion if necessary (Please screenshot before deletion)
    \item \textbf{First Offense}: Formal Warning.
    \item \textbf{Second Offense}: One (1) day ban from Slack.
    \item \textbf{Third Offense}: Seven (7) day ban from Slack.
    \item \textbf{Further Offenses}: Can lead to removal from CSH Slack.
\end{itemize}

\item Harassment

Ex. Unwanted behavior, advances, or requests for favors. Unwelcomed written, visual, or physical conduct (since this deals with Slack, written mostly). Offensive, severe, and/or frequent remarks about a person. Baiting for an argument with an individual or group. Interfering with an individual's right to an education and participation in a program or activity. This definition of Harassment is an extension of the RIT definition of Harassment, as defined in the RIT University Policies (C06.0.III.C)
\begin{itemize}
    \item \textbf{Emergency Triage}: Message Deletion if necessary (Please screenshot before deletion)
    \item \textbf{First Offense}: Formal Warning
    \item \textbf{Second Offense}: Three (3) day ban from Slack, Report to RA.
    \item \textbf{Third Offense}: Ten (10) day ban from Slack, Report to RA.
    \item \textbf{Further Offenses}: Can lead to removal from CSH Slack.
\end{itemize}

\item Racism / Sexism / etc

Ex. Posts made that are discriminatory in nature that create an unwelcoming environment to individual members or groups of members based upon immutable characteristics. 
\begin{itemize}
    \item \textbf{Emergency Triage}: Message Deletion if necessary (Please screenshot before deletion)
    \item \textbf{First Offense}: Formal Warning.
    \item \textbf{Second Offense}: Three (3) day ban from Slack, Report to RA / ResLife.
    \item \textbf{Third Offense}: Ten (10) day ban from Slack, Report to RA / ResLife.
    \item \textbf{Further Offenses}: Can lead to removal from CSH Slack.
\end{itemize}

\item E-Board or RTP abusing admin privileges

Ex. Abuse of any ability afforded a user by having Workspace Admin/Owner status
\begin{itemize}
    \item \textbf{Emergency Triage}: Immediate temporary removal of Administrator Privileges.
    \item \textbf{First Offense}: Formal Warning, removal of Administrator Privileges for one (1) week.
    \item \textbf{Second Offense}: Removal of Administrator Privileges for one (1) month.
    \item \textbf{Third Offense}: Permanent removal of Administrator Privileges.
\end{itemize}

\item Posting obscene images in violation of the Code of Conduct (abuse of House Services)

Ex. images that may upset, trigger, or offend an individual or group. For instance, pornography and gore
\begin{itemize}
    \item \textbf{Emergency Triage}: Message Deletion if necessary (Please screenshot before deletion)
    \item \textbf{First Offense}: Formal Warning.
    \item \textbf{Second Offense}: Seven (7) day ban from Slack.
    \item \textbf{Third Offense}: Seven (7) day ban from Slack.
    \item \textbf{Further Offenses}: Can lead to removal from CSH Slack.
\end{itemize}

\item Issues In Personal DMs

Ex. If there is a problem in conversation in private messages. Messages also may fall under other categories such as Harassment or Racism / Sexism / etc, and may be dealt with as such

CSH E-Board does not have a say in interpersonal issues. For further support, please contact the floor RA/Reslife

\item Abuse of Reporting Mechanisms

Ex. Spamming or baiting performed using the ReportBot Slackbot
\begin{itemize}
    \item \textbf{Emergency Triage}: N/A
    \item \textbf{First Offense}: Formal Warning.
    \item \textbf{Second Offense}: Three (3) day ban from Slack.
    \item \textbf{Third Offense}: Ten (10) day ban from Slack.
    \item \textbf{Further Offenses}: Can lead to removal from CSH Slack.
\end{itemize}

\end{enumerate}

All offenses expire 2 (2) (two) years after the time of offense.
This means if on January 1st 2022 someone gets a written warning and then another one again on January 2nd 2024 the one on 1/2/2024 will act as the first offense again.
This is to prevent someone who did something stupid their first year wont get stuff built up on them their fourth year.
These are suggestions, please note that: severity and frequency do not correlate.
Meaning, if someone makes a particularly egregious harassment violation as a first offense eboard can decide to punish as a "Third violation" level.
And vice versa if someone does 3 minor harassment violations eboard can decide to punish that on a "First violation" level. Make sure anyone who has admin privleges knows these guidlines.
Eboard hands out these punishments but RTPs are allowed to emergency remove something if deemed fit.
Any edits to this document and its meaning may be made at any time by Eboard, but any semantic changes must be logged and presented to house.

\end{document}
